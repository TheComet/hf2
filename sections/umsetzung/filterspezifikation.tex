\subsection{Filterspezifikation}

Das  Stubfilter   ist   durch   die   g-Parameter   des  Prototypfilters,  die
Grenzfrequenz (Passband) $f_C$, die Bezugsfrequenz $f_{\frac{\lambda}{4}}$ und
die Bezugsimpedanz $Z_0$ vollständig spezifiziert.

\paragraph{g-Parameter}
\begin{mdframed}
\begin{equation*} 
\begin{array}{rclcl} 
g_0 & = & 1 \\ 
g_1 & = & 0.973 \\ 
g_2 & = & 1.372 \\ 
g_3 & = & 1.803 \\ 
g_4 & = & 1.372 \\ 
g_5 & = & 0.973 \\ 
g_6 & = & 1 \\ 
\end{array} 
\end{equation*} 
\end{mdframed}

\paragraph{Bezugsgrössen}

\begin{mdframed}
\begin{equation*} 
\begin{array}{rclcl} 
f_C & = & 0.8 GHz \\ 
f_{\frac{\lambda}{4}} & = & 2GHz \\ 
Z_0 & = & 50 \Omega \\ 
\end{array} 
\end{equation*} 
\end{mdframed}

Die  Bezugsgrössen  werden   erst   ab   der  Richards  Frequenztransformation
interessant. Anders sieht es aber  mit  den  g-Parametern  des Prototypfilters
aus.  Diese geben Auskunft über Filtertyp, Stopbandrippel und Sperrbandrippel,
welche   das   Stubfilter   aufweisen   soll.   Erst   mit   Kenntnis   dieser
Filtereigenschaften  kann  beurteilt  werden, ob das synthetisierte Stubfilter
die Filterspezifikationen  einhält.  Deshalb wird in einem ersten Schritt eine
Simulation des  Prototypfilters  mit  den gegebenen g-Parametern durchgeführt.

Abbildung      \ref{fig:graph-LC}       illustriert       das       simulierte
Prototyp-Tiefpass-Filter  5.  Ordnung mit den Elementwerten $g_0$  bis  $g_6$.

\begin{figure}
    \centering
 	\includegraphics[width=\imagewidth]{images/Topologie_Prototyp.png}
 	\caption{Prototyp-Tiefpass-Filter 5. Ordnung}
 	\label{fig:Topologie_Prototyp.png}
\end{figure}

\begin{figure}
    \centering
 	\includegraphics[width=\imagewidth]{images/Ovw_Prototyp.png}
 	\caption{Übersichtsdarstellung S21}
 	\label{fig:Ovw_Prototyp}
\end{figure}

\begin{figure}
    \centering
 	\includegraphics[width=\imagewidth]{images/graph-LC.png}
 	\caption{Detailansicht S21 und S11}
 	\label{fig:graph-LC}
\end{figure}

\begin{figure}
    \centering
 	\includegraphics[width=\imagewidth]{images/filtertypes.png}
 	\caption{Filtertypen, \cite{ref:wikipedia:chebyshev}}
 	\label{fig:filtertypes.png}
\end{figure}

Die    Übersichtsdarstellung     der     Eingügedämpfung     S21    (Abbildung
\ref{fig:Ovw_Prototyp}) zeigt, dass das Filter keinen Stopbandrippel aufweist.
Währenddem  die  Detailanssicht   der   Einfügedämpfung   S21   in   Abbildung
\ref{fig:graph-LC}  einen  kleinen Equirippel von $-0.04058 dB$  im  Passbands
sichtbar macht. Dieses Verhalten trifft auf das eines Chebyshev-Filter des Typ
I  zu,  wie  aus  Abbildung  \ref{fig:filtertypes.png} unschwer erkennbar ist.

Zusammenfassend   k\"onnen   mithilfe   der   Simulation  also  die  folgenden
Filtereigenschaften des Stubfilters gefunden werden:

\begin{mdframed}
    \begin{equation*} 
        \begin{array}{rclcl} 
            Filtertyp & = & Chebyshev\ Typ\ I\\
            Passbandrippel & = & -0.04058 dB \\ 
            Stopbandrippel & = & kein 
        \end{array} 
    \end{equation*} 
\end{mdframed}

Ein verlustloses Filter  lässt  sich  nicht  nur  mit  der  Einfgedämpfung S21
sondern auch mit der Reflexion S11 beschreiben, weil der folgende Zusammenhang
gilt:

\begin{equation}
    {|S11|}^2 + {|S21|}^2 = 1
\end{equation}

Somit  kann  die  Reflexion im Passband mithilfe des Equirippels  im  Passband
berechnet werden.

\begin{equation}
    |S11(f_C)| = \sqrt{1-{|Equirippel|}^2} = -20 dB
\end{equation}

Um das Resultat zu validieren, wurde die Reflexion S11  des  Filters ebenfalls
simuliert   und   in  der  Detailansicht   in   Abbildung   \ref{fig:graph-LC}
dargestellt.

