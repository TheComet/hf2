\subsection{Optimierung}

Im Anschluss an die Synthese wird das Stubfilter optimiert, wodurch die Breite der Stubs angepasst wird. Dabei  werden  die
Schranken  des  Optimieres so gesetzt, dass sich der Sperrbereich  verbessert,
während  der  Equirippel  im   Passband  gleich  bleiben  soll.  In  Abbildung
\ref{fig:vor-nach-optimierung} wird  das  synthetisierte Stubfilter (rot,blau)
dem optimierten Stubfilter (braun, magenta) gegenübergestellt.

Ein Vergleich des  Passbands  zeigt,  dass  sich  die  Filterordung  nach  der
Optimierung  von 5 auf 9 erhöht hat. Dies leuchtet nicht sofort ein, weil  das
Stubfilter vor und nach der Optimierung immer noch aus gleich vielen Elementen
besteht.  Der  Grund  für die Erhöhung der Filterordnung ist, dass  sich  beim
Einfügen   der   Unit-Elementen   vier   zusätzliche   Pol-  und   Nullstellen
eingeschlichen  haben. Die Unit-Elemte sind aber so berechnet, dass  sich  die
zusätzlichen  Pol- und Nullstellen allesamt aufheben. Aus diesem  Grund  ist das Verhalten des Filters
nach der Kuroda Transformation immer noch genau  gleich  wie
nach  der Richard's Frequenztransformation. Nach der  Synthese  beinhaltet
das synthetisierte  Stubfilter  also  vier  redundante Elemente die gar keinen
Einfluss auf das Filterverhalten haben.

Wird  das  Stubfilter   optimiert  so  verschieben  sich  die  vier  Pol-  und
Nullstellen  und sie werden im Passband ersichtlich, wobei  sich  gleichzeitig
die Filterordnung erhöht.


\begin{figure}[h!]
    \centering
    \includegraphics[width=\linewidth]{images/graph-optimierung}
    \caption{Stubfilter vor und nach Optimierung, von Frau Spuhler zur Verfügung gestellt}
    \label{fig:vor-nach-optimierung}
\end{figure}

