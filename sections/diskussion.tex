\section{Diskussion}

In dieser Arbeit wurde aufgezeigt, wie Stubfilter nach der klassischen Methode synthetisiert werden können. Stubfilter die nach dieser Methode synthetisiert werden, beinhalten redundante Elemente und verunmöglichen somit eine optimale Realisation des Filters. Durch den Einsatz von Optimierern können die Filtereigenschaften verbessert werden, indem die zuvor redundaten Elemente auch genutzt werden. Dies führt aber zu einer Erhöhung der Filterordnung und somit zu einer Diskrepanz mit der Filterspezifikation. Zudem erweist sich die klassische Filtersynthese mit anschliessender Optimierung als sehr aufwändig. Deshalb werden heutzutage grösstenteils Synthesetools eingesetzt, welche diese aufwändige Arbeit übernehmen. Die Tools basieren aber immer noch auf den Grundlagen der klassischen Filtersynthese. Aus diesem Grund finden wir es sehr wichtig, dass jeder angehende Elektroingenieur einmal etwas von der klassischen Filtersynthese gehört hat.
