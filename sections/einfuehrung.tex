\section{Einleitung}

In dieser  Arbeit  geht es um die Synthetisierung und Optimierung
eines Mikrowellenfilters (Stubfilter) in Microstrip-Technologie.

Ausgangslage für die Synthethisierung des Stubfilter sind voberechnete ``g-Parameter'', welche nach Methoden der klassischen LC-Filtersynthese gefunden werden. Diese ``g-Parameter'' beschreiben ein impedanznormiertes LC-Prototypfilter auf welches zwei verschiedene  Transformationen
angewendet werden: Die Richard's Transformation und die Kuroda Transformation.

Mit der Richard's Transformation können die konzentrierten Elemente (L,C) des Prototypfilter mit verteilte Elementen (Stubs) ersetzt werden. Nach dieser Transformation spricht man von einem Stubfilter. Das Problem ist aber, dass sich dieses Filter nicht realisieren lässt, weil sich alle Stubs am gleichen Ort befinden. 

Deshalb wird im Anschluss die Kuroda Transformation verwendet,  um eine physikalische Distanz zwischen
den  Stubs  zu  schaffen, damit das Filter auch in Realit\"at umgesetzt  werden  kann. Obwohl für die Kuroda Transformation sogenannte ``Unit Elements'' eingebaut werden müssen, ändert sich die Filterordnung nicht.

Im Anschluss an die Filtersynthese wird der Sperrbereich des Filters optimiert, was eine Erhöhung der Filterordnung um die Anzahl eingebauter ``Unit Elements'' bewirkt. Grund dafür ist blablabla ....Erkenntnisse bla

\newpage

%Die Spulen und Kondensatoren vom  LC-Filter  k\"onnen  mithilfe  der Richard's
%Transformation ganz durch gleich-lange Stubs ersetzt werden (Leitungsst\"ucke,
%die  genau $\frac{\lambda}{4}$ lang sind und entweder am Ende  kurzgeschlossen
%oder offen sind). Geometrisch befinden sich die  Stubs  aber  alle im gleichen
%Ort und das Filter kann noch nicht realisiert werden.




%TODO Richard Transform -> How L/C get transformed to TLSC and TLOC, and how TLIN cannot be transformed
%TODO Unit elements: Dämpfungspol 
