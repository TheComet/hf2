\section{Einf\"uhrung}

In dieser  Arbeit  geht  es um die Entwerfung, Optimierung und Synthetisierung
eines Mikrowellenfilters (Stubfilter) in Microstrip-Technologie.

Der Ansatz st\"utzt auf Wissen zur Entwerfung von Reaktanz-Filter. Spezifisch,
sogenannte  vorberechnete  ``g-Parameter'' werden verwendet, um ein Frequenz--
und Impedanznormiertes LC-Netzwerk zu entwerfen.

Danach  werden  zwei  verschiedene  Transformationen   auf   das   LC-Netzwerk
angewendet: Die Richard's Transformation und die Kuroda Transformation.

Die Spulen und Kondensatoren vom  LC-Filter  k\"onnen  mithilfe  der Richard's
Transformation ganz durch gleich-lange Stubs ersetzt werden (Leitungsst\"ucke,
die  genau $\frac{\lambda}{4}$ lang sind und entweder am Ende  kurzgeschlossen
oder offen sind). Geometrisch befinden sich die  Stubs  aber  alle im gleichen
Ort und das Filter kann noch nicht realisiert werden.

Die Kuroda Transformation wird  verwendet,  um  physikalische Distanz zwischen
den  Stubs  zu  schaffen, damit es auch in Realit\"at umgesetzt  werden  kann.
Dabei  werden  sogenannte  ``Unit Elements''  eingebaut, 


%TODO Richard Transform -> How L/C get transformed to TLSC and TLOC, and how TLIN cannot be transformed
%TODO Unit elements: Dämpfungspol 
