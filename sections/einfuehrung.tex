\section{Einleitung}

In  der  Kommunikationstechnik  werden  oft  Filter  im   Giga-Hertz   bereich
ben\"otigt. Die Realisierung dieser Filter ist  jedoch  auch  mit  den  besten
Kondensatoren und Spulen  wegen  der  Streu-Reaktanzen  nicht  mehr m\"oglich,
weshalb ein anderer Ansatz mit R\"uckgriff auf die Leitungstheorie n\"otig ist.

Das    Ziel    in    dieser   Arbeit   ist   es,   ein   Tiefpass-Filter   auf
Mikrostreifen-Technologie zu dimensionieren.

Wir  verwenden  dabei  verschiedene  Transformationen, um ein LC-Filter in ein
sogenanntes   ``Stub-Filter''   umzuformen.  Durch  Optimierung   werden   die
Charakteristiken des Filters um einiges verbessert.

Die Problematik ist ein Netzwerk zu finden, das auch realisierbar ist. Wie wir
sp\"ater  sehen  werden  k\"onnen Leitungsimpedanzen  nur  in  einem  gewissen
Bereich praktisch erzeugt werden.
