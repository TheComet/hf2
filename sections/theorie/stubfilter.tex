\subsection{Stubfilter}

LC-Filter  können in einem weiten Frequenzbereich  eingesetzt  werden.  Jedoch
wird  bei  konzentrierten Elementen zu höheren Frequenzen hin der Einfluss der
parasitären Eigenschaften immer deutlicher, so dass hohe Anaforderungen an die
Bauteilgüte  gestellt  werden  müssen. Im GHZ-Bereich wird es daher  zunehmend
attraktiv,  statt  konzentrierten  Kapazitäten  und  Induktivitäten  verteilte
Strukturen  in  Form  von  Leitungen  zu  verwenden.  Man  spricht   dann  von
sogenannten Leitungsfilter.
%Quelle direktes Zitat: 
%https://books.google.ch/books?id=MTVQAgAAQBAJ&pg=PA203&lpg=PA203&dq=leitungsfilter+hochfrequenztechnik&source=bl&ots=Ljf0GRJkZt&sig=n-0G9H5VZ9iuPc2qepAATnjA47M&hl=de&sa=X&ved=0ahUKEwj38u-T79DUAhUSL1AKHUsyDNMQ6AEINTAB#v=onepage&q=leitungsfilter%20hochfrequenztechnik&f=false%

Es gibt verschiedene Arten um  Leitungsfilter zu realisieren. Eine Möglichkeit
der  Realisierung  ist das Stubfilter.  Dieses  Filter  verwendet  gleichlange
kurzgeschlossenen Leitungen (TLSC) und  leerlaufende  Leitungen(TLOC), die nur
an einem Ende verbunden werden. Das andere Ende wird  entweder kurzgeschlossen
oder eben offen gelassen. Bei  diesen  Leitungen handelt es sich um sogennante
Stubs(Stichleitungen), die sich bei  hohen  Frequenzen  wie  reaktive Elemente
(L,C)   verhalten   und   somit   die   Realisierung  eines  Mikrowellenfilter
ermöglichen.
%Der grosse Vorteil von Stubfiltern ist, das eine geschlossene Theorie zur Filtersynthese existiert. 

